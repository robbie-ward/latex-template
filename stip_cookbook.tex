% This is a template for rapidly starting new Latex documents.
% Some considerations:
%   - use pdflatex instead of xelatex or lualatex for compatibility with arxiv and publishers
%       - no system fonts, instead use self-contained font packages like crimson and helvet
%       - specify utf-8 input and T1 font encoding explicitly
%   - use natbib/bibtex instead of biblatex/biber for compatibility with arxiv and publishers

%-------------------------------------------------------------------------------------------
% enter the author and paper information below:
%-------------------------------------------------------------------------------------------
\newcommand{\myname}{Robert Ward}
\newcommand{\myemail}{rward48@gatech.edu}
\newcommand{\myaffiliation}{School of Public Policy\\Georgia Institute of Technology}
\newcommand{\paperdate}{November 2019}
\newcommand{\papertitle}{The STIP Cookbook}
\newcommand{\papercitation}{Ward, R. 2019. \papertitle. Under review at \emph{Journal Name}.}
\newcommand{\paperkeywords}{Science of Science, Organization, Networks, Bibliometrics}
\newcommand{\acknowledgments}{Red Bull Energy Drink}

%-------------------------------------------------------------------------------------------
% begin preamble setup
%-------------------------------------------------------------------------------------------
\RequirePackage[l2tabu,orthodox]{nag}   % warn if using any obsolete or outdated commands
\documentclass[12pt,letterpaper]{article} % document style

% import encoding and font packages for pdflatex, in order
\usepackage[T1]{fontenc}                % output T1 font encoding (8-bit) so accented characters are a single glyph
\usepackage[utf8]{inputenc}             % allow input of utf-8 encoded characters
\usepackage{utopia}      %  crimson is other optoin             % document's serif font, in the style of minion pro | crimson
\usepackage[scaled]{helvet}                     % document's sans serif font, helvetica

% import language and regionalization packages, in order
\usepackage[strict,autostyle]{csquotes} % smart and nestable quote marks
\usepackage[USenglish]{babel}           % automatically regionalize hyphens, quote marks, etc
\usepackage{microtype}                  % improves text appearance with kerning, etc

% import everything else
\usepackage{abstract}                   % allow full-page title/abstract in twocolumn mode
\renewcommand{\abstractnamefont}{\bf\sffamily}
\usepackage{authblk}                    % footnote-style author/affiliation info
\usepackage{booktabs}                   % better looking tables
\usepackage{caption}                    % custom figure/table caption styles
\usepackage[final]{draftwatermark}      % watermark paper as a draft
\usepackage{endnotes}                   % enable endnpagesotes
\usepackage{geometry}                   % configure page dimensions and margins
\usepackage{graphicx}                   % better inclusion of graphics
\usepackage{hyperref}                   % hypertext in document
\usepackage{natbib}                     % author-year citations w/ bibtex, including textual and parenthetical
\usepackage{rotating}                   % rotate wide tables or figures on a page to make them landscape
\usepackage{setspace}                   % configure spacing between lines
\usepackage{titlesec}                   % custom section and subsection heading
\usepackage{url}                        % make nice line-breakble urls
\usepackage[svgnames]{xcolor}	% add in extra colors
\usepackage{tikz}
\usetikzlibrary{trees}
\usepackage{multirow}
\usepackage{lscape}

% location of figure files, via graphicx package
\graphicspath{{./figures/}}

% configure the page layout, via geometry package
\geometry{
	paper=letterpaper,         % paper size
	top=3.8cm,                   % margin sizes
	bottom=3.8cm,
	left=3.8cm,					% originally 4cm for left and right
	right=3.8cm}
\setstretch{1.3}              % line spacing originally 1.02
\clubpenalty=10000             % prevent orphans
\widowpenalty=10000            % prevent widows


% set section/subsection headings as the sans serif font, via titlesec package
\titleformat{\section}{\normalfont\large\bfseries\color{black}}{\thesection.}{0.3em}{}
\titleformat{\subsection}{\normalfont\small\bfseries\color{black}}{\thesubsection.}{0.3em}{}

% change title spacing

% page numbering
\pagestyle{empty}

% make figure/table captions sans-serif small font
\captionsetup{font={footnotesize,sf},labelfont=bf,labelsep=period}

% configure pdf metadata and link handling, via hyperref package
\hypersetup{
	pdfauthor={\myname},
	pdftitle={\papertitle},
	pdfsubject={\papertitle},
	pdfkeywords={\paperkeywords},
	pdffitwindow=true,         % window fit to page when opened
	breaklinks=true,           % break links that overflow horizontally
	colorlinks,						% add link color
	citecolor = interceptpurple, % add link color
	urlcolor = interceptpurple, % add url color
	linkcolor = Black,
	%colorlinks=false,          % remove link color
	pdfborder={0 0 0}          % remove link border
}

\definecolor{interceptpurple}{rgb}{0.4, 0.32, 1.0}

% configure watermark appearance: to turn it on/off use [final] argument at package import
\SetWatermarkText{Draft}
\SetWatermarkScale{1.3}
\SetWatermarkLightness{0.90}


%-------------------------------------------------------------------------------------------
% end preamble setup, begin document
%-------------------------------------------------------------------------------------------
\begin{document}

\title{\textbf{\papertitle}}
\author[]{Robert Ward\footnote{Email: \href{mailto:\myemail}{\myemail}}}
%\author[2]{Other Fine Authors}
\affil[]{\myaffiliation}
%\affil[2]{Other Fine Institutions}
%\date{\paperdate}


\maketitle\thispagestyle{empty}

\section{What we do}
At STIP, the general mission is to understand science, technology and innovation by analyzing documents.





\section{Robbie's Interpretation of the Research Strategy}
STIP has been around for 15 years (I think) with what seems to be a similar model along the way.  From an organizational perspective this is extremely successful.  Part of the reason why is that STIP is very good at executing a single research strategy.  I see the elements as this.  (1) Diversified funding across different sectors are diverse from NSF to DoD, other research centers and universities, and firms.  The research is focused on the problems that these funders have.  Generally we are not operating at the cutting edge, methodologically.  The strategy instead, is to clearly define an applied problem that the funder or their stakeholders have.  Often this problem is to understand what technologies/topics are important or emerging today and what technologies/topics will be important in 5 or 10 years.  So the problem is both a lack of knowledge and about reducing uncertainty about what will happen in the near future.  (2) So we are primarily doing applied research though there is some development of tools.  This focus means that we are often not operating at the cutting edge, methodologically.  Having an 80\% solution where we understand the advantages and disadvantages, and how it will behave on different data is generally more important at STIP than using or developing cutting edge methods.  And it turns out very simple methods are often very effective.  So when you are thinking about the analysis, it's useful to think about what is the simplest 80\% method I can use, rather than what is the maximally effective, or most elegant method.  It is additionally very important that you can explain the methods to both technical and nontechnical clients who are rarely experts in our field.  So you might explain to someone who has a lot of subject matter expertise in a technological domain, or alternatively to someone who doesn't (e.g. the army futures guys) but who will understand the technical aspects if you explain, statistically, what is going on.  One of the implications of having diverse funders is that you need to be able to phrase these explanations in many different ways.  (also one of the great things about being a PhD student at STIP is that you can learn how to do this).

When faced with a hiccup or challenge, say the intermediate results look mushy or some parts don't quite fit or seem a bit suspect.  The STIP strategy (all my interpretation) is to narrow the question and analysis (within the needs of the funder) and focus on getting a clean, solid, explainable answer to the narrow question.  This isn't universal, in some groups the response would be to start adding many more things to fill the gaps.

As a PhD student (in my experience) there will be a mix of projects where the analysis is fully specified in advance (like the tech landscaping, or emergence scoring).  And projects where you will have autonomy at the methodological level to do what you think makes sense.  When you have autonomy, a few things to keep in mind.  (1) Jan, Phil and the rest of the group don't need to know how the estimator or how the math underneath the method works.  Just that it works.  But it is important that you know that it works or doesn't.  (2) You need to be able to explain what the product will look like, very clearly.  This is the data, this is the data source and these are the fields and how each variable is made/found.  You need to be able to explain the result that your analysis is going to produce, specifically and justify that this answers the question.  When explaining the method, it is only to justify why doing it this way will give you the result you need at a conceptual level.  You need to understand the mechanics, how your method works, but the rest of the group usually doesn't beyond the conceptual level.

Lastly, STIP puts pressure on you to ship, to get things out the door.  Operationally, it is important to pick and choose which projects you want to change course on.


\section*{Theory}
Nullam consequat ex lectus. Aliquam sagittis lacus quis dui tincidunt, eget accumsan dolor dictum. Duis vel ultricies sem. Aliquam dapibus tellus consequat, ornare diam vitae, dapibus dui. Maecenas blandit posuere augue quis ornare. Fusce posuere tortor fringilla lacus viverra, a iaculis quam vehicula. Aenean convallis tellus ut leo egestas, ac consequat nisi efficitur.
\clearpage
\section*{Reading List}
Most important stuff is in section 2.1, everything else is a bonus.  
\subsection{Introductions}
Pre-bootcamp and early semester readings to help you get up to speed.  Chapters 1-3 in \cite{sugimoto2018measuring} give a gentle introduction to bibliometric/scientometric data, concepts and indicators.  Then read Chapter 2 in \cite{latour_laboratory_1979} to understand how and why documents are produced by scientists, what they can and can't tell us about the underlying science and technology.  After \textit{Laboratory Life}, read \cite{fortunato_science_2018} for a brief overview of recent work in the Science of Science.

For a more general introduction to computational social science, read Chapters 1-3 of \citep{salganik2019bit}.

\subsection{STIP-Specific}
A few other readings will be useful once we get into the bootcamp.  The samples of documents that we analyze tend to be drawn similarly.  There is (1) A set of inclusion and exclusion terms that are applied to the data and/or metadata of a corpus.  The documents that are selected with this become the sampling frame. (2) There is a data source, typically a database like Web of Science or Scopus for papers or PatStat for patents.  The accessible population is the set that contains all of the documents in all of these databases.  

Typically we will only use one or two different databases for each type of document because there is a lot of overlap between say Web of Science and Microsoft Academic.  (3) We generally draw a non-probability sample whose boundaries are defined by the search strategy (1), the data source (2) and the degree to which the sampled documents have missing data fields.  Documents are also called records.  To understand how that search strategy gets made, skim \cite{porter2008refining}.

If you have time, skim \cite{rotolo_what_2015}, mostly for section 4.  It's a bit long but lays out the definition for an emerging technology that we use.  I don't think you're going to work much on the technological emergence research but good to know anyways.  

\subsection{Anticipating your projects}
From what I understand, the project you are going to start on is about measuring research performance at a multi-university research center.  We'll be working together on it so you aren't going into the deep end right away.  \cite{leahey_prominent_2017} and \citep{dahlander_ties_2013} will be good places to start, \cite{biancani_superstars_2018} might be interesting too.  Read \cite{leahey_prominent_2017} if you have time before the bootcamp (will probably be in mid-January), it will be helpful for understanding how the indicators that \cite{sugimoto2018measuring} discuss can be used as part of a cohesive study.  No big deal if not.

I'll get pdf's together for you and throw them in a Dropbox folder.  Don't worry about the technical aspects yet, I'll cover that in the bootcamp. 


%  Another thing to start looking at is the networks literature \citep{newman_coauthorship_2004} is a gentle introduction to this.  His textbook is excellent as well \cite{newman2018networks}. Chapters 3,4,6-10 in the second edition will be helpful to reference if you have questions.  Plenty of deeper dives but Newman is a good place to start.


% print the footnotes as endnotes, if any exist
\IfFileExists{\jobname.ent}{\theendnotes}{}

% print the bibliography
\setlength{\bibsep}{0.00cm plus 0.05cm} % no space between items
\bibliographystyle{apalike}
\bibliography{references}

\end{document}
